% Tento soubor nahraďte vlastním souborem s obsahem práce.
%=========================================================================
% Autoři: Michal Bidlo, Bohuslav Křena, Jaroslav Dytrych, Petr Veigend a Adam Herout 2019
\chapter{Úvod}

V dnešní době, kdy lidé jsou zvyklí na větší pohodlí než kdysi, narůstá i počet chytrých domácností. Chytré domácnosti by měly usnadnit život lidem, okolí, i dokonce celé planetě. Když se řekne chytrá domácnost, může se někomu například vybavit jakýsi robot, který vám uklidí, či dokonce přinese nákup.
Chytrá domácnost může být jakkoliv chytrá, dokonce se může podobat i domácnostem z různých sci-fi filmů. Můžeme si uvést příklad z každodenního života většiny z nás. Večer
přijdete domů s těžkým nákupem, musíte vyndat z kapsy klíče (se štěstím, že jsou
právě v kapse), pracně otevřít, položit nákup, rozsvítit na chodbě, zhasnout na chodbě,
rozsvítit v kuchyni atd.
Teď si představte, že se blížíte ke dveřím, kamera rozpozná váš
obličej, otevřou se dveře, rozsvítí se světla, vozík odveze váš nákup k lednici a Vy si nemusíte ničeho všímat.

Chytrá domácnost má ale i další výhody a to především při úspoře energie. Úspora energie hraje v dnešní době velice důležitou roli. Při výrobě elektrické energie se vyprodukuje mnoho skleníkových plynů, které zatěžují životní prostředí.
Chytrá domácnost může rapidně snížit produkci těchto plynů. Největším odběratelem elektrické energii v domácnostech jsou především různá topná telěsa ať už v podobě elektrického topení, ohřívačů vody, osvětlení apod.
Tito největší odběratelé pracují většinou celý den a komplexně pro celý dům, avšak určitě změny by byly velice přínosné, např. vytápění pouze obývaných prostor, snižování intenzity osvětlení podle okolní intenzity světla, různé akce vykonané podle aktuálního času apod.
Podle mého názoru se chytré domácnosti dostanou do popředí, na trhu v oblasti informačních technologiích, velice rychle. Je samozřejmé, že chytré domácnosti a chytré zařízení existují.
Ovšem nastává problém, kdy tato zařízení jsou velice drahá a výrobce, většinou, podporuje pouze svá zařízení a je téměř nemožné sestavit chytrou domácnost z různých komponentů od různých výrobců.

Cílem této práce je vytvořit kompletní řešení chytré domácnosti, které bude levné a připojená různá zařízení budou mezi sebou komunikovat bez ohledu na výrobce.
Dále hlavním rysem tohoto projektu je poskytnutní přístupnosti k domácnosti. Tento systém je navržen tak, aby mohl být nasazen v cloudu a být přístupný pro tisíce uživatelů zároveň. Vyžaduje se určitá robustnost systému a hlavně bezpečnost.
Někteří uživatelé však cloudové služby nepodporují a bezpečnost je u nich na prvním místě. Pro tento případ je systém navržen tak, aby mohl běžet na samostatném mini počítači pouze v rámci domácnosti. Dalším cílem práce je poskytnout uživateli
intuitivní rozhraní pro snadné zprovoznění chytré domácnosti a následné správě domácnosti.

V kapitole \ref{terminy} najdete vše, co je potřeba k porozumnění zbytku práce. V další kapitole \ref{navrh} je pospán návrh daného řešení, v kapitole \ref{implementace} implementace řešení, v kapitole \ref{testovani} popsáno testování a v poslední řadě v kapitole \ref{zaver} závěr práce.

\chapter{Důležité termíny a technologie}
\label{terminy}

V této kapitole jsou popsané důležité termíny, technologie, nástroje a další, co je nezbytné k porozumnění zbytku práce. Nejdůležitější je pochopit koncept \emph{IoT}, který je popsán v podkapitole \ref{terminy:iot}.
V dalších podkapitolách najdeme stručné vysvětlení pojmu \emph{MCU - neboli mikrokontrolér}(kapitola \ref{terminy:iot}), informace o různých zařízeních, která jsou obsažena v implementaci této chytré domácnosti. Specifičtěji se jedná hlavně o moduly \emph{ESP} (kapitola \ref{esp}) a mini počítač \emph{Raspberry Pi} (kapitola \ref{raspberry}).
Komunikaci mezi zařízeními je pomocí protokolu \emph{MQTT}, komunikace s uživatelským rozhraním (dále jako \emph{frontend}) a serverem, kde je obsažena celá business logika (dále \emph{backend}), zajištuje protokol \emph{HTTP}.

\bigskip

\section{IoT - Internet věcí (Internet of Things)}
\label{terminy:iot}

V informatice se jedná o označení pro síť fyzických zařízení, vozidel, domácích spotřebičů a dalších zařízení, která jsou vybavena elektronikou,
softwarem, senzory, pohyblivými částmi a síťovou konektivitou, která umožňuje těmto zařízením se propojit a vyměňovat si data.
Každé z těchto zařízení je jasně identifikovatelné díky implementovanému výpočetnímu systému,
ale přesto je schopno pracovat samostatně v existující infrastruktuře internetu.~\cite{wiki:iot}

Jak vyplývá z definice \emph{IoT}, zařízení v domácnosti jsou propojena (např. přes switch, nebo router) a navzájem komunikují. Po síti se posílají data z různých senzorů a různá zařízení na tato data reagují jinak.
Data mohou být vyhodnocována přímo daným zařízením, nebo systémem, který je pro zařízení dostupné. Tento systém může být ve formě distribuované sítě, kde data mohou být vyhodnocována a zasílána i na druhý konec světa.
V \emph{IoT} velmi často vystupuje \emph{umělá inteligence} (dále jako AI - artificial intelligence), která může předávat data různým zařízením a vyhodnocovat pomocí neuronových sítí.

\emph{IoT} je využíváno i v prostředí různých výrobních procesů, či linek. V továrnách kde velké stroje plní svoji práci, mohou také mezi sebou komunikovat a rozdělení práce mezi dané stroje může být tímto přístupem velice efektivní.
Stroje rozloží výrobní proces tak efektivně, že všechny stroje jsou zatížené tak, jaké požadavky jsou zavedené na daný proces. Amortizace těchto strojů se sníží a proto i náklady na výrobu jsou nižší. Stroje mohou mezi sebou komunikovat dokonce
i mimo lokální továrny a vyměňovat si data např. o počtu materiálu, který je přítomen na skladě a opět efektivně přerozdělovat material do více skladů a továren.

\newpage

\section{Vestavěný systém (Embedded system)}
\label{terminy:vestaveny}

Vestavěný systém, neboli \emph{Embedded system} je hardwarový systém, který je tzv. \emph{microprocessor-based}(založený na mikroprocesoru).
Obsahuje software, který je navržen k vykonávání určité specifické činnosti, buď jako nezávislý systém, nebo jako část většího celku.
Nejvýznamnější komponent vestavěného systému je integrovaný obvod, který je navržen pro operace, které jsou závislé na reálném čase(dále jako \emph{real-time} systém).~\cite{embedded:info}

K vestvěnému systému lze připojit mnoho vstupních/výstupních periferních zařízení, které spolupracují s mikrokontrolérem obsaženým ve vestavěném systému.
Na obrázku níže je vyznačeno jednoduché schéma, které pojednává o asociaci mezi vestavěným systémem, vstupních/výstupních zařízení a dalších náležitostí.

Nejdůležitější částí je připojení vstupních zařízení, mezi které může patřit např. klávesnice, tlačítka a různé senzory. Vestavěný systém pomocí mikrokontroléru vyhodnotí získaná data
a přivede na výstup patřičnou hodnotu, odpovídající programovému zpracování dat podle obsaženého softwaru (také označovaný jako \emph{firmware}). Mezi výstupní zařízení můžeme zahrnout např. různá relé, tranzistory, audio a video výstupy a mnoho dalších.

\begin{figure}[hbt]
  \centering
  \includegraphics[width=.6 \linewidth]{obrazky-figures/embedded_system.png}
  \caption{
    Koncept vestavěného systému
  }
  \label{figure:embedded}
\end{figure}

\newpage
\section{MCU - mikrokontrolér (Microcontroller unit)}
\label{terminy:mcu}

Mikrokontrolér, nebo také jednočipový počítač je integrovaný obvod.
Mikrokontroléry jsou velice spolehlivé, kompaktní a proto hojně využívány k řízení různých elektronických systému, obykle pomocí mikroprocesoru, paměti a různých periferních zařízení.
Tato malá zařízení jsou optimalizována pro vestavěné systémy popsané výše, které potřebují zpracovat interakci s digitálními, analogovými, nebo elektromechanickými komponenty.

Název pro mikrokontrolér byl velice vhodně zaveden, protože opravdu zdůrazňuje charakteristiku typu daného zařízení. Mikrokontroléry jsou malá zařízení, která dokáží velké věci.
Mikrokontroléry hrají důležitou roli v technologické budoucnosti. Získavají si velikou oblibu nejen u různých SW\footnote{\textbf{SW} - software}/HW\footnote{\textbf{HW} - hardware} expertů, ale i studentů a spousty hobby nadšenců.~\cite{mcu:info}
Díky mikroprocesorům lze jednoduše propojit softwarový nástroj, či aplikaci s hardware komponenty.

Blížící se budoucnost přinese spoustu nových možností v oblasti miktrokontrolérů
a počet zařízení obsahující mikrokontrolér může v blízké době přesáhnout i několika desítek miliard kusů. Výkon u těchto zařízení bude vyšší a velikost menší, což je občas velice důležitý faktor na poli chytrých domácností.
V chytré domácnosti jsou zařízení sestavena s pomocí mikrokontroléru, který se stará o danou specifickou funkci. V této bakalářské práci jsou především použity mikrokontroléry s čipem ESP-8266 od firmy \emph{Espressif Systems}
a minipočítač \emph{Raspberry Pi}. Obrázek níže obsahuje odpovídající architekturu skoro každého mikrokontroléru.
Mezi hlavní prvky patří analogově-digitální převodník, čítače, časovače, modul řídící PWM\footnote{\textbf{PWM} - pulse-witdth modulation (pulsně šířková modulace). }

\begin{figure}[hbt]
  \centering
  \includegraphics[width=.6 \linewidth]{obrazky-figures/mcu.jpeg}
  \caption{
    Koncept mikrokontroléru
  }
  \label{figure:mcu}
\end{figure}

\section{Moduly obsahující čip ESP8266}
\label{terminy:esp8266}

\emph{ESP8266} je levný WiFi mikročip, který s přidáním několika komponentů je schopen plnohodnotně plnit úlohu mikrokontroléru. Tento mikročip je velice oblíbený v oblasti chytrých zařízení díky své stabilititě, výkonosti, velikosti, ceně a mnoho dalšího.~\cite{wiki:esp}
Na trhu existuje spousty modulů, které obsahují právě zmíněný mikročip a plní tak úlohu mikrokontroléru. Mezi nejoblíbenější moduly, které obsahují tento mikročip jsou \emph{ESP-01}, který obsahyje pouze 2 GPIO\footnote{\textbf{GPIO} - univerzální vstupní/výstupní pin (General-purpose input/output)} piny, poté větší \emph{Wemos D1 mini} a \emph{NodeMCU}.
Tento projekt chytré domácnost je sestaven hlavně z těchto modulů díky svým přednostem a využití. Pro maximálně 2 jednotlivé I/O\footnote{\textbf{I/O} - vstup/výstup (\textbf{I}nput/\textbf{O}utput)} periferie připojené k modulu je využit modul \emph{ESP-01}. Pro komplexní využítí a připojení více periferií je použit modul \emph{Wemos D1 mini}.

\begin{figure}[hbt]
  \centering
  \includegraphics[width=.2 \linewidth]{obrazky-figures/esp_standalone.png}
  \caption{Samostatný ESP8266 čip}
  \label{figure:esp8266}
\end{figure}

\section{Minipočítač Raspberry Pi}
\label{terminy:raspberry}

\emph{Raspberry Pi} je v informatice označení pro jednodeskový počítač, který je se svým výkonem srovnatelný se slabším stolním počítačem. Obsahuje video a audio výstupy,
ethernetový port, USB porty a výstup na dedikovaný monitor určený přímo pro \emph{Raspberry Pi}. Tento malý počítač, rozměry podobný kreditní kartě, je využit v této bakalářské práci na nasazení služeb, kdy uživatel nechce využívat cloudové služby.
Modelů \emph{Raspberry Pi} je na trhu více, kde se liší svým výkonem, velikostí, nebo kompatibilitou s různými rozhraními. Za účelem nasazení služeb v chytré domácnosti lze využít i menšího sourozence z rodiny počítačů \emph{Raspberry Pi} a to přesně \emph{Raspberry Pi W},
kde vytvořené, nebo poskytnuté služby jsou optimalizované pro běh na zařízeních s menším výpočetním výkonem a operační pamětí.

\begin{figure}[ht]
  \centering
  \includegraphics[width=.5 \linewidth]{obrazky-figures/raspberry.png}
  \caption{Raspberry Pi 4}
\end{figure}

\newpage

\section{MQTT protokol}
\label{terminy:mqtt}

\emph{MQTT} neboli \emph{Message Queuing Telemetry Transport} protokol je jeden z nejvýznamnějších komunikačních protokolů IoT zařízení a systémů.
\emph{MQTT} patří do kategorie aplikačních protokolů, kde jeho hlavní předností je malá velikost datové hlavičky a možnost komunikovat v sítích s omezenou propustnosí.
Patří do skupiny \emph{Publish-Subscribe}(dále Pub-Sub) protokolů. Hlavním principem \emph{Pub-Sub} protokolů je výměna zpráv mezi dvěma typy účastníků. První z nich je tzv. odebíratel(dále \emph{subscriber}),
který se přihlašuje k odběru zpráv s daným temátem(dále \emph{topic}).
Subscriber může přijímat několik zpráv s různými tématy a v průběhu se od odběru i odhlašovat. Druhým typem je tzv. vydavatel(dále \emph{publisher}), který posílá zprávy do určitého
topicu.~\cite{mqtt:info}

\subsection{MQTT Broker}
\emph{MQTT Broker} je služba (nebo-li software, který běži v cloudu, či lokálním PC), která se stará o rozesílání a spravování zpráv, které různá zařízení posílají. MQTT Broker je prostředník mezi účastníky typu \emph{subscriber} a \emph{publisher}.
Každé zařízení, které chce využívat MQTT protokol a komunikovat s ostatními zařízeními, musí definovat URL brokeru, přes který budou tyto zprávy procházet.
Jak již bylo řečeno, MQTT je Pub/Sub protokol, proto když přijde, od účastníka typu publisher, zpráva ke specifickému topicu, musí se podívat, kdo všechno je přihlášen k odběru daného topicu a rozeslat všem účastníkům, typu subscriber, tuto zprávu.
Každé zařízení může jak přijímat, tak i odesílat zprávy k danému topicu.~\cite{wiki:mqtt_broker} V této bakalářské práci je využit \emph{MQTT Broker} s názvem \emph{HiveMQ}.

\subsection{Zabezpečení}
V \emph{IoT} je velice důležitá i bezpečnost a pomocí \emph{MQTT Brokeru} lze dosáhnout určitého bezpečnostního cíle. Je opravdu nevítané, aby se někdo cizí dostal bez autentizace k odběru vašich
zpráv procházející přes \emph{MQTT Broker} a mohl zasílat zprávy do různých dalších zařízení připojená v domácnosti.
Útočník tak má šanci kontrolovat celou vaši domácnost. Každý \emph{MQTT Broker} od různých výrobců disponuje různou sadou zabezpečení a vlastností.~\cite{wiki:mqtt_broker}

Možné typy zabezpečení:
\begin{itemize}
  \item Využívání šifrovaného portu \textbf{8883} namísto nešifrovaného \textbf{1883}
  \item Autentizace pomocí uživatelského jména a hesla
  \item \textbf{TLS}\footnote{\textbf{TLS} - Transport layer security} připojení
  \item \textbf{OAuth}\footnote{\textbf{OAuth} - Poskytuje autentizační a autorizační správu} správa
\end{itemize}
\subsection{QoS - Kvalita služeb (Quality of services)}
Každá zpráva může specifikovat, jak bude zajištěna kvalita služeb u poslání daných zpráv.~\cite{wiki:mqtt_broker}
\begin{itemize}
  \item \textbf{QoS 1} - Nejvýše jednou; zpráva je zaslána pouze jednou bez potvrzení.
  \item \textbf{QoS 2} - Alespoň jednou; zpráva stále zasílána dokud nedorazí potvrzení o přijetí.
  \item \textbf{QoS 3} - Přesně jednou; zajištění, že přijemce dostane zprávu pouze jednou.
\end{itemize}

\newpage
\subsection{Typy zpráv}
\emph{MQTT} protokol využívá sadu různých typů zpráv pro vlastní běh. Na obrázku \ref{figure:mqtt_flow}lze zahlédnout posloupnost zpráv.~\cite{wiki:mqtt_broker}
Mezi základní typy zpráv u protokolu MQTT patří:
\begin{itemize}
  \item \textbf{CONNECT} - slouží pro ustanovení připojení.
  \item \textbf{CONNACK} - (\emph{Connection acknowledge}) typ zprávy je odezva z brokeru klientu, který požadoval o připojení a vytvoří se spojení mezi uzly.
  \item \textbf{SUBSCRIBE} - neboli odběr; klient pošle tuto zprávu brokeru na určitý topic, u kterého chce přijímat dané zprávy
  \item \textbf{UNSUBSCRIBE} - slouží k odhlášení klienta od odebírání zpráv z určitého topicu
  \item \textbf{PUBLISH} - slouží na publikování zpráv na různý topic (např. zasílání dat ze senzorů)
  \item \textbf{DISCONNECT} - slouží k odhlášení klienta z celého systému
\end{itemize}

\begin{figure}[ht]
  \centering
  \includegraphics[width=.7 \linewidth]{obrazky-figures/mqtt_flow.png}
  \caption{Posloupnost MQTT zpráv}
  \label{figure:mqtt_flow}
\end{figure}


\newpage
\section{Aplikační prostředí}
\label{terminy:app_prostredi}
V této kapitole jsou k dispozici informace o technologiích, nástrojích, či specifikacích, které jsou využity v této bakalářské práci pro vytvoření backend aplikace.
Backend aplikace v případě této bakalářské práce se zabývá samotnou logikou celého systému, vykonáváním požadavků na databázi, poskytnutím výpočetních sil pro složitější operace a požadavky od zařízení v domácnosti.
Vytváří rozhraní, ke kterému může přistupovat spousta klientských aplikací a spravovat tak celý management systému chytré domácnosti.

Jedná se tak o jádro celého systému.
Hlavním úkolem této backend aplikace je zprostředkovat služby, pro zařízení chytré domácnosti, pro připojení do systému. Ukládá důležité hodnoty ze zařízení, co se týče stavů výstupních zařízení a poslední hodnoty vstupních zařízení, aby byly stále dostupné při připojení klientské aplikace.
Klientské aplikace vytváří požadavky na tuto backend aplikaci a získávájí tázaná data. Tato aplikace je vytvořena robustnějším způsobem, aby byla schopna obstarávat požadavky několika tisíců uživatelů.
K dané aplikaci se váže i otázka bezpečnosti.

Uživatel se musí autentizovat v daném systému, aby bylo možno identifikovat přihlášeného uživatele a s tím související autorizační práva na různé zdroje informací.
O autentizaci a správu uživatelů se, pro větší bezpečnost, stará služba \emph{Keycloak}. Jádrem této backend aplikace je technologie \emph{Quarkus}.

\subsection{Quarkus}
\label{app_prostredi:quarkus}

\begin{figure}[!ht]
  \centering
  \includegraphics[width=.45 \linewidth]{obrazky-figures/quarkus_logo.png}
  \caption{Quarkus logo}
  \label{figure:quarkus_logo}
\end{figure}

V dnešní době, kdy žijeme v době cloudových služeb, IoT a open-source projektů přichází do popředí kontejnery(\emph{containers}), mikroslužby(\emph{microservices}), reaktivní programování, cloud-native aplikace a spousty dalšího.
Díky těmto novým architekturám, nástrojům, přináší vývoj aplikací jiný rozměr a to speciálné větší produktivitu a výkonnost aplikací. Programovací jazyk \emph{Java} už je na trhu přes 20 let a stálé zůstává mezi nejpopulárnějšími programovacími jazyky na světě.
V informačních technologiích se však technologie a architektury mění velice rychle a je téměř nemožné využívat technologie přes 20 let bez větších změn. Aplikace v kontejnerech by měly mít co nejmenší velikost, rychlý start při restartu a být škálovatelné.
To však v případě dosavadních Java aplikací nebylo až tak možné. Framework \emph{Quarkus} by však měl vše změnit.

\textbf{Quarkus} je Kubernetes\footnote{\textbf{Kubernetes} - orchestrace kontejnerů na úrovni OS} Native Java framework, který nese označení \emph{Supersonic Subatomic Java} (nadzvuková subatomární Java). Tento framework je přímo ušitý pro GraalVM\footnote{\textbf{GraalVM} - generuje nativní kód} a HotSpot(klasická JVM\footnote{\textbf{JVM} - \textbf{J}ava \textbf{V}irtual \textbf{M}achine}).
Je složen z dostupných Java knihoven a standardů, které patří mezi ty nejlepší svého typu. Vyznačuje se termínem zvaným \emph{Container First}, kde celý framework je založený na nasazení aplikací v kontejnerech a dále v cloudových službách.~\ref{quarkus:infoDev}
\newpage

Na obrázku \ref{figure:quarkus_stats} je porovnání frameworku \emph{Quarkus} s tradičním \emph{Cloud-Native} prvkem a dále porovnání vytvoření spustitelného souboru do nativního kódu pomocí \emph{GraalVM} a klasické využítí JVM pomocí \emph{OpenJDK}\footnote{\textbf{OpenJDK} - Open \textbf{J}ava \textbf{D}evelopment \textbf{K}it}.
Hodnoty jsou určeny pro klasickou \emph{REST}\footnote{\textbf{REST} - \textbf{RE}presentational \textbf{S}tate \textbf{T}ransfer (architektura rozhraní)} architekturu rozhraní a dále s přidanými operacemi \emph{CRUD}\footnote{\textbf{CRUD} - \textbf{C}reate, \textbf{R}ead, \textbf{U}pdate, \textbf{D}elete (Vytvořit, Číst, Aktualizovat, Smazat)}.
První horizontální polovina obrázku se zabývá pamětí celého programu a závislostí. Oproti tradičnímu \emph{Cloud-Native} prvku je program vygenerovaný do nativního kódu až 10x menší. V \emph{Cloud-Native} aplikací je to opravdu znatelný rozdíl.
V druhé polovině je k zahlédnutí nastartování celé aplikace a první odpověď ze serveru při požadavku. Rapidní snížení rychlosti oproti klasickému tradičnímu \emph{Cloud-Native} prvku a v tom dokáže být \emph{Quarkus} rychlejší více než 250x.~\ref{quarkus:website}

\begin{figure}[hbt]
  \centering
  \includegraphics[width=1 \linewidth]{obrazky-figures/quarkus_stats.png}
  \caption{Porovnání frameworku Quarkus s alternativami}
  \label{figure:quarkus_stats}
\end{figure}

\subsection{Vert.x}
\label{app_prostredi:vertx}
\begin{figure}[hbt]
  \centering
  \includegraphics[width=.30 \linewidth]{obrazky-figures/vertx.png}
  \caption{Vert.x logo}
\end{figure}
\emph{Vert.x} je sada nástrojů pro vytvoření reaktivních aplikací, kde hlavní předností je malá velikost, nízké nároky na výpočetní prostředky a rozsáhlý ekosystém.
Samotný \emph{Quarkus} je založený na \emph{Vert.x} technologii a skoro všechny síťové featury v technologii \emph{Quarkus} závisí na technologii \emph{Vert.x}.
Tato technologie přináší mnoho zajímavých vylepšení, jako jsou neblokující požadavky na server, jednoduchá souběžnost procesů(\emph{Concurrency}), nebo například podpora několika programovacíh jazyků.~\ref{wiki:vertx}

\newpage

\subsection{Hibernate ORM}
\label{app_prostredi:hibernate}
\begin{figure}[hbt]
  \centering
  \includegraphics[width=.2 \linewidth]{obrazky-figures/hibernate.png}
  \caption{Hibernate logo}
\end{figure}

\emph{Hibernate ORM} (dále pouze jako \emph{Hibernate}) je ORM\footnote{\textbf{ORM}- \textbf{O}bjektově \textbf{R}elační \textbf{M}apování} framework vyvíjen společností \emph{Red Hat}.
Poskytuje možnost mapování objektově orientovaného modelu do relační databáze. Vývojář tak není zatížen vytvářením mezivrstvy, které konvertuje objekty a vytváří SQL dotazy na databázi.
Vývojář tak může definovat tabulku databáze dvěma způsoby:
\begin{enumerate}
  \item Pomocí XML\footnote{\textbf{XML} - \textbf{E}xtension \textbf{M}arkup \textbf{L}anguage (značkovací jazyk)} mapovaní, kdy vývojář vytvoří XML soubor s definovanými atributy tabulky a jejich omezeními.
  \item Vývojář pouze vytvoří POJO\footnote{\textbf{POJO} - \textbf{P}lain \textbf{O}ld \textbf{J}ava \textbf{O}bject (obyčejný \emph{Java} objekt)}, kde definuje tabulku v relační databázi. Pomocí anotací se označí atributy třídy, které se namapují do tabulky databáze.
        Tímto způsobem lze vytvořit několik tabulek v databázi a lze i pomocí anotací definovat asociaci mezi tabulkami a kardinalitu asociací.
\end{enumerate}
\emph{Hibernate} zprostředkuje rozhraní, které lze využít pro dotazy na databázi. Vývojář může vytvořit vlastní metody, u kterých lze definovat dotaz na databázi pomocí HQL\footnote{\textbf{HQL} - \textbf{H}ibernate \textbf{Q}uery \textbf{L}anguage (dotazovací jazyk podobný SQL)} jazyka,
který je jednodušší a bližší samotným třídám (označovaným jako \emph{entity}).

\subsection{OIDC}
\label{app_prostredi:oidc}

\begin{figure}[hbt]
  \centering
  \includegraphics[width=.35 \linewidth]{obrazky-figures/OIDC.png}
  \caption{OIDC logo}
\end{figure}

\textbf{OIDC} nebo-li \emph{OpenID Connect} je autentizační protokol, založený na specifikaci \emph{OAuth 2.0}. Tento protokol využívá služba \emph{KeyCloak}, která je použita v této bakalářské práci.
\emph{OIDC} protokol využívá \emph{JSON Web Tokens} (dále jako \emph{JWT}), což je formát využit pro reprezentaci ID tokenu. \emph{OIDC} je hlavně o autentizaci uživatelů.
Jeho účelem je poskytnou pouze jeden přihlašovací proces pro více webových stránek.~\ref{terminy:oidc}

\newpage
\subsection{KeyCloak}
\label{app_prostredi:keycloak}

\begin{figure}[hbt]
  \centering
  \includegraphics[width=.3 \linewidth]{obrazky-figures/keycloak2.png}
  \caption{Keycloak logo}
\end{figure}

\textbf{KeyCloak} je open-source služba, která se stará o autentizaci a autorizaci uživatelů a další předností je \emph{Identity and Access Management}(Management přístupu a identit), kde administrátor může upravovat práva uživatelů, přidávat do skupin s určitými právy apod.
Tato služba je podporovaná firmou \emph{Red Hat} a několik let už je v popředí popularity služeb, které se starají o bezpečnost webových aplikací. \emph{KeyCloak} disponuje rozsáhlou komunitou vývojářů, kteří jsou velice aktivní v příspívání do daného komunitního produktu.
\emph{KeyCloak} obsahuje spousty nových vylepšení a je opravdu velice dobře škálovatelný(\emph{scalable}) a upravovatelný(\emph{customizable}).

Obsahuje spousty adaptérů pro klientské aplikace, tudíž nezáleží v takové míře na programovacím jazyce.
Využívá standardní zabezpečovací protokoly, mezi které patří \emph{OIDC}, \emph{OAuth 2.0}\footnote{\textbf{OAuth 2.0} - autorizační protokol} a \emph{SAML 2.0}\footnote{\textbf{SAML 2.0} - \textbf{S}ecurity \textbf{A}ssertion \textbf{M}arkup \textbf{L}anguage(autentizační/autorizační protokol)}.
Lze dokonce využít i asociované autentizační a autorizační služby třetích stran pro udělení přístupu uživateli, registraci a mnoho dalšího. Lze získat přístup i od tzv. \emph{Social providers}(poskytovatelů sociálních sítí) jako jsou např. \emph{Facebook}, \emph{Google}, \emph{Twitter} apod.

Tento produkt využívám ve své práci pro autentizační služby a správu uživatelů. Vývojář se tak nemusí starat o perzistenci uživatelů a jejich správu. O vše se stará služba \emph{KeyCloak}.
\emph{KeyCloak} poměrně dobře znám, protože jsem jedním z aktivních přispěvovatelů do produktu.

\newpage
Na obrázku \ref{figure:keycloak_flow} je popsán autentizační proces uživatele. V prvním kroku uživatel žádá o zdroj informací u aplikace, která je asociována s \emph{KeyCloak} aplikací (v mém případě s mojí \emph{backend} aplikací).
Aplikace pošle na službu \emph{KeyCloak} autentizační požadavek, služba se podívá, zda už je uživatel autentizován a když ano, pošle aplikaci odpověď o úspěšném přihlášení a aplikace povolí uživateli přístup k datům.

V opačném případě je uživateli poskytnut seznam poskytovatelů služeb starající se o identitu uživatelů.
Uživatel si může podle definovaného autentizačního toku (\emph{Authentication flow}) vybrat jakým způsobem se autentizuje. Když vše proběhne v pořádku, \emph{KeyCloak} pošle odpověď aplikaci o úspěšném přihlášení a uživatel dostane přístup k datům.

\begin{figure}[hbt]
  \centering
  \includegraphics[width=.8 \linewidth]{obrazky-figures/keycloak_flow.png}
  \caption{Autentizační tok \emph{KeyCloak}}
  \label{figure:keycloak_flow}
\end{figure}

\newpage
\section{JSON}
\label{terminy:json}

\textbf{JSON} nebo-li \textbf{J}ava\textbf{S}cript \textbf{O}bject \textbf{N}otation je formát pro strukturovaná data. Formát dat se sestavuje podle dvojice atribut-hodnota a případně i pole.
\emph{JSON} je nezávislý na programovacích jazycích a je odvozen, jak název vypovídá, z programovacícho jazyka \emph{JavaScript}. \emph{JSON} je velice rozšířený ve spustě aplikací, kde téměř všude nahrazuje \emph{XML}.

Tento formát zápisu dat je použit v systému, který popisuje tato bakalářská práce, ve dvou odlišných případech. První z nich je při komunikaci s klientskou aplikací, kde se data zprostředkovávají pomocí tohoto formátu \emph{JSON}.
V druhém případě je tento formát použit i u \emph{MQTT} zpráv posílané ze zařízení. Jedná se hlavně o zprávy, které se posílají pomocí \emph{MQTT} protokolu na správu zařízení v domácnosti.
Na obrázku ~\ref{figure:json} je vidět ilustrační zápis dat pomocí \emph{JSON} formátu.

\begin{figure}[hbt]
  \centering
  \includegraphics[width=.6 \linewidth]{obrazky-figures/json_example.png}
  \caption{Ukázka \emph{JSON}}
  \label{figure:json}
\end{figure}

\section{RESTful služba}
\label{terminy:restful}
\todo{COMPLETE}

\chapter{Současný stav a návrh řešení}
\label{navrh}

V této kapitole je shrnut současný stav řešení chytré domácnosti od nejvýznamnějších firem, které se zabývají tímto tématem a samotný návrh řešení, který by měl danou problematiku vyřešit.
Jak již bylo popsáno v úvodu, největším úskalím řešení chytré domácnosti je celková cena, a kompatibilita se zařízeními od různých výrobců.

V kapitole \ref{navrh:existujici} jsou popsány existující řešení chytré domácnosti od různých firem a v kapitole \ref{navrh:reseni} vlastní řešení problému a schopnost dosažení požadovaných cílů.

\section*{Existující řešení}
\label{navrh:existujici}

Kompletní řešení chytré domácnosti existuje v dnešní době už opravdu nepřeberné množství. Podle mého názoru se postupem let dostane do popředí ještě více \emph{IoT} zařízení a systémů kompletní chytré domácnosti.
Už v dnešní době spousta obrovských IT firem, mezi které patří např. \emph{Samsung}, \emph{LG}, \emph{Apple} a \emph{Google}, vyvíjí takové systémy.

Například firma \emph{Samsung} představila na trh produkty s názvem \emph{SmartThings}, které lze využít pro sestavení chytré domácnosti.
Hlavním prvkem, který potřebujete je tzv. \emph{Hub}, který slouží jako mozek celého systému.
U svých zařízení většinou využívají pro komunikaci protokol zvaný \emph{Zigbee}, který se řadí do tzv.\emph{ad-hoc}\footnote{\textbf{ad-hoc} - decentralizovaná bezdrátová síť} sítí.
Signály ze zařízení dosahují vzdálenosti 10 až 20 metrů.
\emph{Samsung} se pyšní tím, že jejich zařízení jsou kompatibilní s více než 100 zařízeními od různých výrobců, ale jejich cena je obrovská.

Lze si to představit na jednoduchém příkladu u chytrého tlačítka od firmy \emph{Samsung}. Toto tlačítko se vyznačuje tím, že dokáže po nastavení ovládat několik světel současně, spotřebiče, apod.
Tlačítko se v průměru prodává za nějakých 15̈́\$, což je v přepočtu něco okolo 380kč. Takové tlačítko, které dokáže sadu podobných, dokonce i stejných úkonů, lze sestrojit pro můj vlastní návrh chytré domácnosti asi za cenu okolo 30kč.

Dalším příkladem může být chytrá zásuvka, kterou \emph{Samsung} prodává za cenu blízkou 35\$, což je v přepočtu asi 880kč a ve své vlastní domácnosti lze sestrojit podobný kus asi za 60kč.
Samozřejmě lze brát v úvahu i to, že \emph{Samsung} zaručuje kvalitu a dlouhodobou spolehlivost svých zařízení, využívá odlišný protokol, ale cena, podle mého názoru, je až přehnaně vysoká.

\newpage

\section*{Obecný návrh řešení}
\label{navrh:reseni}

Návrh vlastního řešení systému chytré domácnosti je poněkud komplikovanější záležitost, díky své rozmanitosti.
Vývojář tak musí být obeznámen s technologiemi a principy z různých odvětví aplikačního vývoje.
Systém chytré domácnosti by měl disponovat velikou škálou možností a být jednoduše škálovatelný.
Pro návrh takového systému je potřeba v první řadě vytvořit analýzu požadavků.
\newline
\newline
Hlavní aspekty, které by měly být splněny:
\begin{itemize}
  \item \textbf{Cena pořízení} - Kompletní systém i jednotlivé komponenty.
  \item \textbf{Zabezpečený systém} - Na všech dostupných úrovních (server, databáze, klientská aplikace, zařízení, přenos dat).
  \item \textbf{Intuitivnost a přístupnost} ovládacích prvků
  \item \textbf{Jednoduchost} - Vytvoření domácnosti a inicializování/připojení zařízení i příliš \textbf{netechnicky zdatným uživatelem}.
  \item \textbf{Sdílení domácnosti} - Schopnost sdílet domácnost s více uživateli.
  \item \textbf{Autorizace uživatelů v domácnosti} - Každý uživatel v různých domácnostech může mít různá přístupová práva.
  \item \textbf{Dostupnost} - Dostupnost systému jak v případě pouze lokální síťě, tak i s pomocí cloudových služeb.
  \item \textbf{Použitelnost} - Schopnost systému obstarávat několik tisíc uživatelů zároveň.
  \item \textbf{Customizace\footnote{\textbf{Customizace} - úprava dle požadavků uživatele} zařízení} - Přidávání/odebírání schopností zařízení jednoduchým způsobem.
\end{itemize}
Jak již bylo řečeno, když na návrh systému chytré domácnosti je pouze jeden vývojář, musí se orientovat v rozmanité sféře technologií, návrhových vzorů apod.
V dnešní době jsou tyto typy vývojářů označovány jako \emph{Full Stack Developers}, kteří vyvíjí \emph{backend}, tak i \emph{frontend} část aplikace.

V kapitole \ref{navrh:backend} bude detailněji popsán \textbf{návrh serverové(\emph{backend}) části} systému, která je nejspíše nejkomplikovanější a nedůležitější ze všech odvětví této bakalářské práce.
Dále v kapitole \ref{navrh:databaze} lze nalézt popis \textbf{návrhu relační databáze} a související věci.
V kapitole \ref{navrh:frontend} je \textbf{návrh klientské aplikace}(\emph{frontend}), kde hlavním jádrem věci je rozložení a vzhled stránky.
Samozřejmě je zde i popis struktury uložení stavu aplikace a operací provádených na pozadí.
V poslední kapitole \ref{navrh:hardware}je \textbf{návrh HW zařízení}, struktura programu.

\newpage
\section{Návrh serverové části}
\label{navrh:backend}
Z hlediska vývoje aplikací je z mého pohledu tato část práce nejkomplikovanější a zároveň nejobtížnejší.
Vývojář musí mít dobrý přehled o technologiích, návrhových vzorech, které lze vhodně aplikovat na danou aplikaci.
Struktura programu této části by měla být napsána přehledně a stylem takovým, aby bylo možno jednoduše rozšiřovat.
Požadavky na aplikaci se mohou velice rychle měnit a vývojář musí dokázat bez většího úsilí implementovat tyto změny.
Serverová část(dále jako \emph{Server}) aplikace se stará o požadavky, které klientská aplikace, potažmo zařízení, posílá na daný server.

Mezi velikou škálou programovacích jazyků, které jsou vhodné na vývoj serverových aplikací, jsem si vybral programovací jazyk \emph{Java},
který se v popularitě těchto technologií drží na předních místech. Programovací jazyk \emph{Java} je objektově orientovaný jazyk, který se velkým způsobem zasadil do vývoje enterprise aplikací.
Jazyk \emph{Java} disponuje rozshálým ekosystémem a existuje spoustu výborných frameworků. Pro účely vývoje serverové aplikace byl využit \emph{Java} framework zvaný \emph{Quarkus}(detailní popis \ref{app_prostredi:quarkus}).

Rozhodl jsem se, že serverová část bude sestrojena jako \emph{RESTful}(podrobněji \ref{terminy:restful}) webová služba komunikující nad protokolem \emph{HTTP},
kde u serveru nebude žádné uživatelské rozhraní, ale pouze programové, známe jako API\footnote{\textbf{API} - \textbf{A}pplication \textbf{P}rogramming \textbf{I}nterface (rozhraní pro programování aplikací)}.
Server poskytne \emph{API}, které využívají klientské aplikace, nebo další služby pro komunikaci se serverem.
Server je bezstavový, což znamená, že si server neukládá stav o požadavcích a každý požadavek je přijímán stejně, bez jakýkoliv priorit, či upřednostnění díky danému stavu uživatele v systému.

\subsection*{Zabezpečení}
\label{backend:bezpecnost}
Každý uživatel, který pošle požadavek na server, musí být autentizován.
Server je asociován s autentizační službou a při každém požadavku na server přepošle tzv.\emph{token}\footnote{\textbf{Token} - žeton pro přístup ke zdrojům informací} autentizační službě
a ta vyhodnotí, zda je uživatel autentizován, potažmo autorizován získat informace z daného endpointu\footnote{\textbf{Endpoint} - HTTP path (cesta požadavku definována pomocí API)}.
Pokud uživatel není autentizován a nevlastní žádaný \emph{token}, musí poslat požadavek na autorizační službu s danými \emph{credentials}\footnote{\textbf{Credentials} - poveření v různých formách (heslo, PIN, OTP, biometrika,...)},
kde získá daný token pro přístup ke zdrojům informací serveru. Token je zasílán v HTTP hlavičce v atributu \emph{Authorization}.

\subsection*{Přístupová práva}
\label{backend:prava}
Přístupová práva pro uživatele jsou různá. Každý uživatel může mít odlišné role v různých domácnostech, tudíž nemá specifikovanou jednu roli jako to občas bývá zvykem u serverových aplikací.
Každá role má různé vlastnosti a práva přístupu k informacím, či práva k vykonání určitých operací.
\newline
Mezi tři základní skupiny se řadí:
\begin{itemize}
  \item \textbf{Administrátor}
  \item \textbf{Klasický uživatel}
  \item \textbf{Dítě}
\end{itemize}

Tyto typy uživatelů jsou hierarchicky rozpoložené, kde uživatel ve skupině \emph{Dítě} je na nejnižším místě s nejmenším počtem přístupových práv, poté \emph{Klasický uživatel} a nejvýše položený je \emph{Administrátor}, který má největší počet přístupových práv.
Tyto základní skupiny jsou dále modifikovatelné. Administrátor domácnosti může vytvořit další podřazené skupiny u kterých definuje přístupová práva a může přidat uživatele do skupin.
Uživatelé ve skupině, které administrátor domácnosti přidělí právo spravovat uživatele ve skupinách, jsou schopni spravovat uživatele ve skupinách pouze v podřazených skupinách.

Uživatel, který si vytvoří vlastní domácnost se automaticky stává administrátorem dané domácnosti a může ji plně spravovat, dokonce může přidávat a odebírat uživatele z domácnosti.
Uživatel, který má minimální roli \emph{Klasický uživatel} v domácnosti je schopen si vytvořit vlastní pokoj, který může plně spravovat. Uživatel s rolí \emph{Dítě} takovou možnost nemá.
Administrátor má také právo poslat pozvánku uživateli do domácnosti s určenou rolí v domácnosti. Uživatel může buď přijmout pozvánku a být tak zařazen do domácnosti s určitou rolí, nebo v opačném případě odmítnout a smazat pozvánku.
\newline

Na obrázcích níže lze nalézt grafické znázornění operací, které lze vykonávat s určitou základní rolí.
Grafické znázornění je vytvořeno pomocí \emph{Use Case}\footnote{\textbf{Use Case diagram} - diagram případů užití} diagramu, který spadá do kategorie diagramu chování definovaný v \emph{UML}\footnote{\textbf{UML} - \textbf{U}nified \textbf{M}odeling \textbf{L}anguage (grafický jazyk pro vizualizaci)}.
\emph{Use Case} diagram zachycuje pouze vnější pohled na modelovaný systém a nepoukazuje na způsob implementace daných operací.~\ref{use_case:info}
Na obrázku \ref{figure:use_case_dite} je zachyceno rozhraní pro uživatele s rolí \emph{Dítě}, na obrázku \ref{figure:use_case_uzivatel} pro uživatele s rolí \emph{Klasický uživatel} a na posledním obrázku \ref{figure:use_case_admin} pro uživatele s rolí \emph{Administrátor}.

\begin{figure}[hbt]
  \centering
  \includegraphics[width=0.4 \linewidth]{obrazky-figures/raspberry.png}
  \caption{Diagram užití pro roli \textbf{Dítě}}
  \label{figure:use_case_dite}
\end{figure}

\begin{figure}[hbt]
  \centering
  \includegraphics[width=0.4 \linewidth]{obrazky-figures/raspberry.png}
  \caption{Diagram užití pro roli \textbf{Klasický uživatel}}
  \label{figure:use_case_uzivatel}
\end{figure}

\begin{figure}[hbt]
  \centering
  \includegraphics[width=0.4 \linewidth]{obrazky-figures/raspberry.png}
  \caption{Diagram užití pro roli \textbf{Administrátor}}
  \label{figure:use_case_admin}
\end{figure}

\subsection*{Správa požadavků ze zařízení}
\label{backend:mqtt}

Důležitou součástí serverové části systému je správa požadavků ze zařízení asociovaných s domácností. Komunikace mezi zařízením a serverem probíhá pomocí protokolu \emph{MQTT}.
Pro každou domácnost existuje na serveru jedna instance \emph{MQTT} klienta. U každé domácnosti lze definovat pouze jeden \emph{MQTT Broker} a náležitá zařízení se připojují přímo k němu.
Daná instance \emph{MQTT} klienta je přihlášená k odběru celého provozu dané domácnosti. Lze tak jednoduše odchytávat komunikaci mezi zařízeními v domácnosti za účelem správy perzistentního obsahu zařízení.
Analýza provozu a komunikace u zařízení slouží k vytváření statistik posledních hodnot, které mohou být použity k efektivnímu chodu domácnosti a její správě.
Lze tak efektivně řídit provoz elektricky náročných zařízení. Dále je možnost analyzovat data, která jsou dále zpracována pomocí většího výpočetního výkonu serveru, než u daných zařízení a generovat tak přislušný výstup.

Pomocí protokolu \emph{MQTT} jsou spravovány i požadavky, týkající se \emph{CRUD} operací daného zařízení. Tyto operace pomocí \emph{MQTT} jsou pouze přístupné zařízením a ne klientským aplikacím.
Zařízení je tak schopno zaslat požadavek na server, který spojí dané zařízení s domácností, či připojení již existujícího zařízení a následně aktivovat dané zařízení pro přístup z klientských aplikací.

\section{Návrh databáze}
\label{navrh:databaze}

Návrh databáze úzce souvisí se serverovou částí systému. Server vytváří rozhraní, pomocí kterého lze vytvářet operace nad danou databází.
Server disponuje určitým \emph{ORM} frameworkem, který mapuje objektově orientový model do relační databáze a vývojař tak není nucen pracovat s databází na nižší úrovni např. pomocí SQL jazyka.

Návrh databáze u daného řešení chytré domácnosti je modelován pomocí \emph{ER diagramu}\footnote{\textbf{ER Diagram} Entity-Relationship diagram(entitně vztahový diagram)}, kde jsou znázorněné entity\footnote{\textbf{Entita} - věc schopná samostatné existence} modelující tabulky databáze a jejich propojení.
Databáze je tak modelována pomocí modelovací jazyka s názvem \emph{UML}. Diagram obsahuje název entity, primární a cizí klíče, atributy entity a asociace tabulek s určitou kardinalitou.
\newpage
Mezi základní entity řešení chytré domácnosti patří:
\begin{itemize}
  \item \textbf{\emph{User}} - základní informace o uživateli
  \item \textbf{\emph{Home}} - domácnost
  \item \textbf{\emph{Room}} - místnost/pokoj
  \item \textbf{\emph{Device}} - zařízení
  \item \textbf{\emph{Capability}} - schopnost zařízení
\end{itemize}

\subsection*{Entita \emph{User} - uživatel}
\label{databaze:user}
Entita \textbf{\emph{User}}(uživatel) obsahuje dva identifikátory. První z nich je primární klíč celé entity a druhý identifikátor typu \emph{UUID}\footnote{\textbf{UUID} - \textbf{U}niversally \textbf{U}nique \textbf{I}dentifier(Univerzální unikátní identifikátor)} slouží k identifikaci uživatele z \emph{JWT} tokenu.
Tato tabulka je pouze pomocnou entitou v udržování informací o uživatelích. Primárním uložištěm informací je autentizační služba, která dále obsahuje i přístupová data.
Daná entita se využívá hlavně v případě, kdy uživatel pro každou domácnost disponuje různými rolemi. Tato entita je dále využívána v identifikaci uživatele v pozvánkách do domácnosti.

\subsection*{Entita \emph{Home} - domácnost}
\label{databaze:home}
Nejdůležitějším celkem celé databáze je entita \textbf{\emph{Home}}, která obsahuje základní potřebné informace o domácnosti. Jako každá tabulka databáze musí mít přiřazený primární klíč s jednoznačným identifikátorem dané entity.
Dále obsahuje prvek \emph{name}, která je typu \emph{String}(řetězec znaků) a je zde uložen název domácnosti. Důležitým atributem entity je \emph{brokerURL}, což je URL \emph{MQTT} brokeru. Domácnost může mít pouze jeden \emph{MQTT Broker}, tudíž stačí uložit jeden prvek s typem \emph{String}.
Entita \emph{Home} dále obsahuje cizí klíč pro entitu \emph{MQTTClient}, kde kardinalita daného vztahu je 1:1 díky tomu, jak již bylo zmíněno, může být pouze jeden \emph{MQTTBroker} a \emph{MQTTClient} pro domácnost.
U klienta se rozumí spíše klient, který spravuje a analyzuje zprávy z komunikačního protokolu \emph{MQTT}.

\subsection*{Entita \emph{Room} - místnost}
\label{databaze:room}
Entita \emph{Home} je provázána s entitou \textbf{\emph{Room}}. Domácnost může obsahovat více místností, ale daná místnost může být zahrnuta pouze v jedné domácnosti. Dále je provázána s entitou \emph{Device}(zařízení), kde představuje seznam zařízení, které nejsou zatím přiřazeny do určitého pokoje a jsou proto tzv. \emph{unassigned}(nepřiřazené).
Dále se k entitě \emph{Home} vztahuje i entita \emph{HomeInvitation}(pozvánky do domácnosti), kde jsou ve vztahu 1:N kvůli tomu, že domácnost může obsahovat několik pozvánek, ale v pozvánce může být zahrnut pouze jedna domácnost.
Jako další provázána s \emph{Home} entitou je entita \emph{User}, která je ve vztahu M:N. Tento vztah představuje asociaci, kde domácnost může obsahovat více uživatelů a uživatelé mohou být obsažení ve více domácnostech.
Vytváří se tak jedinečná asociace v celé databázi. V poslední řadě si uchovává množinu uživatelů, kteří tuto místnost vlastní a mají určitá vyhrazená práva k místnosti.

\subsection*{Entita \emph{Device} - zařízení}
\label{databaze:device}
Podstatná entita systému je \textbf{\emph{Device}}, která obsahuje opět jednoznačný identifikátor, název zařízení a status, zda je zařízení aktivní.
Zařízení může být provázáno s místností, nebo i s domácností a proto tudíž obsahuje dva cizí klíče referující na \emph{Home} a \emph{Room}.
Pokud zařízení nemá přiřazeno žádnou místnost, automaticky je vložen do množiny, kterou spravuje entita \emph{Home} a obsahuje nepřiřazená zařízení.
Každé zařízení je lehce modifikovatelné a uživatel si může vytvořit své vlastní zařízení disponující určitými tzv.\emph{capabilities}(schopnostmi), jako je např. měření teploty, vlhkosti, ovládání světelných zařízení atd.

\subsection*{Entita \emph{Capability} - schopnost zařízení}
\label{databaze:capability}
Entita podřazená zařízení nese název \textbf{\emph{Capability}}. Tato entita obsahuje prvky, které jsou nejblíže samotnému HW zařízení.
Obsahuje identifikátor, název, příznak \emph{enabled}(zda je schopnost povolená), typ dané schopnosti a pin ke kterému je připojena komponenta zprostředkovávající schopnost připojena.
Kardinalita vztahu mezi entitou \emph{Device} a příslušnou entitou je v poměru 1:N, kde zařízení může mít několik schopností a daná jednoznačná schopnost může být přiřazena jednomu zařízení.

\begin{figure}[hbt]
  \centering
  \includegraphics[width=1 \linewidth]{obrazky-figures/raspberry.png}
  \caption{ER Diagram databaze}
  \label{figure:er_databaze}
\end{figure}

\newpage
\section{Návrh klientské aplikace}
\label{navrh:frontend}

\section{Návrh zařízení}
\label{navrh:hardware}

\todo{Use case}